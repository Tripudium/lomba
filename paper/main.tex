%\documentclass{article}
\documentclass[11pt,a4paper]{amsart}

\usepackage{geometry}
\geometry{a4paper,total={150mm,237mm},left=30mm,top=30mm}

%\usepackage{titlesec}

\usepackage{amssymb}
\usepackage{amsthm}
\usepackage{amsmath}
\usepackage{mathdots}
\usepackage{graphicx}
\usepackage{xcolor}
\usepackage{xfrac}
\usepackage{bm}

\usepackage{mathtools}
\usepackage{leftindex}

\input{macros}

\title{Continuous-Time Limit Order Book Forecasting with Mamba State Space Models}

\author{Francesco Papini}
\address{(Papini) Warwick Mathematical Institute, University of Warwick, UK}
\email{martin.lotz@warwick.ac.uk}
\author{Martin Lotz}
\address{(Lotz) Warwick Mathematical Institute, University of Warwick, UK}
\email{martin.lotz@warwick.ac.uk}


\date{\today}
\begin{document}
\maketitle

\begin{abstract}
We study continuous-time forecasting of limit order book event streams using a selective state-space 
backbone (Mamba) coupled with a mixture-of-exponentials temporal point-process head and hierarchical mark 
decoders. The model produces calibrated probabilities for near-horizon arrivals and structured distributions 
over event marks (price move, size, side, event type, book level). 
We describe the data preprocessing, binning, hierarchical smoothing, and training objectives needed 
to make state-space models competitive on noisy market message streams, and we outline an evaluation protocol. 
\end{abstract}


\tableofcontents

\section{Introduction}

Modern electronic markets produce high-frequency streams of limit order book (LOB) messages: submissions, cancellations, and executions arriving at millisecond granularity. 
Trading systems increasingly need calibrated forecasts of both \emph{when} the next event will arrive and \emph{what} its attributes (marks) will be, in order to manage inventory, 
routing, and queue position. While discrete-time sequence models capture local patterns, continuous-time structure drives queue dynamics, especially around sharp changes in activity.

This paper introduces a continuous-time forecasting architecture that couples a selective state-space backbone (Mamba, \citep{gu2023mamba}) with a mixture-of-exponentials temporal point-process (TPP) head 
and hierarchical decoders for event marks (price move, size, time bucket, event type, side, and book-level proxy). 

The key ingredients are:
\begin{itemize}[leftmargin=*,itemsep=0.25em]
    \item \textbf{Continuous-time modeling}: A mixture-of-exponentials intensity yields closed-form survival and arrival probabilities over arbitrary horizons, enabling calibrated risk estimates.
    \item \textbf{Selective state-space backbone}: Mamba processes long streams efficiently and can incorporate both inter-arrival times and absolute time-of-day signals.
    \item \textbf{Hierarchical marks}: Coarse/residual binning with neighbor-aware smoothing produces sharp yet stable distributions over marks, aligning with microstructure semantics (e.g., tails beyond the spread).
    \item \textbf{Practical training recipe}: Quantile-based bin fitting on the training set, censor-aware losses, cosine warmup with AdamW, and optional soft binning for heavy-tailed features.
\end{itemize}

We focus on specifying the model, data processing, and training objectives. Experimental results and ablations will be added when broader evaluations across instruments and market regimes are complete.

\section{Preliminaries}

\subsection{Limit order book dynamics}
A limit order book (LOB) is an electronic record of buy and sell queues indexed by price levels. Let $b_k(t)$ 
and $a_k(t)$ denote queue sizes at the $k$-th level at time $t$from the best bid and best ask, respectively, 
with best prices $p^{\text{bid}}(t)$ and $p^{\text{ask}}(t)$ such that $p^{\text{bid}}(t) < p^{\text{ask}}(t)$. 
Events update the state of the LOB via:
\begin{itemize}[leftmargin=*,itemsep=0.25em]
    \item \textbf{Limit add}: insert volume $s$ at price $p$ on side $\sigma \in \{\text{buy},\text{sell}\}$, increasing the corresponding queue: $b_k \gets b_k + s$ or $a_k \gets a_k + s$ when $p$ matches level $k$.
    \item \textbf{Cancel/amend}: remove or reduce standing volume at a given level.
    \item \textbf{Marketable order}: consume volume against the opposite queue. If volume 
    sweeps multiple levels, best prices update. Trades are recorded as executions.
\end{itemize}
The midprice is $m(t) = \frac{1}{2}\bigl(p^{\text{bid}}(t) + p^{\text{ask}}(t)\bigr)$ and the spread 
is $p^{\text{ask}}(t) - p^{\text{bid}}(t)$. Queue evolution is piecewise-constant with jumps at event 
times; see \citep{gould2013limit,bouchaud2008markets,cont2010stochastic,abergel2016limit} for formal treatments. 
Most electronic venues operate price-time priority: orders are first ranked by price (best bid/ask at the front), 
and within each price level by arrival time. 

A market maker maintains resting bids and asks, earning the spread while bearing inventory and 
adverse-selection risk. Key controls include (i) \emph{quote placement} (choosing price levels and sizes 
relative to the spread and queue depth), (ii) \emph{quote lifetime} (cancelling or repricing as flow risk 
changes), and (iii) \emph{inventory targets} (skewing quotes or size to neutralize exposure). 
Short-horizon forecasts of arrival intensity and mark distributions are central to these decisions.

\subsection{Temporal point processes}
We model the arrival times of LOB events using temporal point processes (TPPs). Let $\{t_i\}_{i \ge 1}$ be a strictly increasing sequence of event times. The history up to time $t$ is denoted by $\mathcal{H}_t = \{(t_i, m_i) : t_i < t\}$, comprising the times and marks $m_i$ (e.g., price, size, type) of all events occurring before $t$.
A TPP is fully characterized by its conditional intensity function $\lambda(t \mid \mathcal{H}_t)$, which represents the instantaneous rate of a new event arrival given the history:
\[
\lambda(t \mid \mathcal{H}_t) = \lim_{\Delta t \to 0^+} \frac{P(N(t + \Delta t) - N(t) = 1 \mid \mathcal{H}_t)}{\Delta t},
\]
where $N(t)$ counts the number of events up to time $t$.
Given the most recent event at $t_n$, the probability density function of the next inter-arrival time $\Delta t = t_{n+1} - t_n$ is
\[
f(\Delta t \mid \mathcal{H}_{t_n}) = \lambda(t_n + \Delta t \mid \mathcal{H}_{t_n}) \exp\!\left(-\int_{t_n}^{t_n + \Delta t} \lambda(u \mid \mathcal{H}_u)\,du\right).
\]
For a mixture-of-exponentials intensity with non-negative weights $\alpha_k$ and rates $\beta_k$,
\begin{equation}
\lambda(\Delta t) = \sum_{k=1}^K \alpha_k \beta_k \exp(-\beta_k \Delta t), \qquad
\Lambda(\Delta t) = \sum_{k=1}^K \alpha_k \bigl(1 - e^{-\beta_k \Delta t}\bigr)/\beta_k,
\label{eq:moe-intensity}
\end{equation}
which yields closed-form survival $S(\Delta t) = \exp(-\Lambda(\Delta t))$ and horizon probability $P(\Delta t \le \tau) = 1 - \exp(-\Lambda(\tau))$. This tractability motivates the head used in our architecture.

\subsection{State-space sequence models}
Continuous-time linear state-space models (SSMs) define a latent $x(t) \in \mathbb{R}^d$ driven by input $u(t)$:
\begin{align}
\dot{x}(t) &= A x(t) + B u(t), \\
y(t) &= C x(t) + D u(t),
\end{align}
with learned $A,B,C,D$. Discretization with step $\Delta$ yields a recurrence $x_{n+1} = \bar{A} x_n + \bar{B} u_n$ and output $y_n = C x_n + D u_n$, equivalent to a convolution with a kernel derived from $(\bar{A},\bar{B},C,D)$. This gives linear-time, memory-efficient inference for long sequences, in contrast to the quadratic cost of attention.

Mamba \citep{gu2023mamba} is a \emph{selective} SSM: the effective transition and input matrices are modulated by the current token, gating information flow based on content while retaining the fast scan/convolution implementation. The result combines long-context capacity with low latency and footprint, making it suitable for high-frequency LOB streams where both history length and computational budget are critical.

\section{Mixture-of-exponentials temporal point processes}
\label{sec:moe-tpp}

Mixtures of exponentials offer a tractable family for modeling inter-arrival densities while retaining flexibility beyond a single-scale Poisson process. Let $K$ denote the number of components. Given non-negative weights $\alpha_k$ (not necessarily normalized) and rates $\beta_k>0$, the intensity and cumulative intensity after the last event are
\begin{align}
\lambda(\Delta t) &= \sum_{k=1}^K \alpha_k \beta_k \exp(-\beta_k \Delta t), \\
\Lambda(\Delta t) &= \sum_{k=1}^K \alpha_k \bigl(1 - e^{-\beta_k \Delta t}\bigr) / \beta_k.
\end{align}
The survival and horizon probabilities follow immediately:
\[
S(\Delta t) = \exp\bigl(-\Lambda(\Delta t)\bigr), \qquad
P(\Delta t \le \tau) = 1 - \exp\bigl(-\Lambda(\tau)\bigr).
\]
This yields closed-form log-likelihoods for observed arrivals ($\log \lambda - \Lambda$) and for censored observations ($-\Lambda$), making the family attractive for maximum-likelihood training and calibration at arbitrary horizons.

\paragraph{Connections to prior work.}
Mixture intensities and hazard mixtures have long been used in survival analysis as semi-parametric approximations \citep{daley2003introduction}. In neural point process literature, mixture or basis expansions of intensities provide a balance between flexibility and closed-form integration (e.g., \citealp{du2016recurrent,mei2017neural,tan2021relaxing}). Our choice mirrors this design: the model learns $(\alpha_k,\beta_k)$ from the Mamba hidden state via linear projections followed by softplus activations (to ensure positivity), preserving analytical integrals, inverse transform sampling, and efficient evaluation of $P(\Delta t \le \tau)$ needed for trading risk controls.

\section{Model}

Our architecture combines engineered representations for LOB messages with a Mamba backbone and specialized output heads 
for time and marks.

\subsection{Feature extraction and binning}
We load raw LOB messages (time, type, order id, size, price, direction) and compute derived features $x_i$ 
for each event $i$:
\begin{itemize}[leftmargin=*,itemsep=0.25em]
    \item \textbf{Price change}: $\Delta p_i = (p_i - p_{i-1}) / \delta$, where $\delta$ is the tick size. This is discretized 
    into a centered window $[\ell, h]$ (e.g., $[-2, +2]$ ticks) plus two tail bins for larger moves.
    \item \textbf{Size and Time}: Log-transformed size $\log(1 + s_i)$ and inter-arrival time 
    $\log(1 + \Delta t_i / \tau_{\text{scale}})$, where $\tau_{\text{scale}}$ is the median positive inter-arrival 
    time in the training set.
    \item \textbf{Categorical}: Event type (limit, cancel, deletion, execution), side (ask/bid), and a level proxy 
    derived from $|\Delta p_i|$ (at-spread vs.\ away).
    \item \textbf{Time-of-day}: Absolute time $t_i^{\text{abs}}$ (seconds from midnight) is encoded cyclically as 
    \begin{equation*}
    \begin{aligned}
    [\sin(2\pi h/24), \cos(2\pi h/24), \sin(2\pi m/60), \cos(2\pi m/60)]
    \end{aligned}
    \end{equation*}
\end{itemize}

For continuous features (size, time), we employ a soft binning strategy to preserve local information. Given bin edges 
$e_0, \dots, e_B$, we compute the distance of a value $x$ to bin centers $c_j = (e_j + e_{j+1})/2$. A softmax with 
temperature $T$ yields weights $w_j \propto \exp(-|x - c_j|/T)$, which are used to compute a weighted sum of learnable bin 
embeddings $E \in \mathbb{R}^{B \times d}$:
\[
\mathbf{e}_{\text{soft}}(x) = \sum_{j=1}^B w_j(x) \mathbf{E}_j.
\]
This allows the model to interpolate between bins, mitigating boundary effects inherent in hard quantization.

\begin{table}[h]
\centering
\caption{Example of raw LOB messages and corresponding derived features.}
\label{tab:feature-example}
\small
\begin{tabular}{lcccccc}
\hline
\textbf{Field} & \textbf{Time} & \textbf{Type} & \textbf{Side} & \textbf{Price} & \textbf{Size} \\
\hline
Raw Value & $34200.123$ & 1 (Limit) & -1 (Ask) & $100.05$ & $100$ \\
Derived Feature & $t^{\text{abs}} = 34200.123$ & Type Code 0 & Side Code 0 & $\Delta p = 0$ ticks & $\log(1+s) \approx 4.62$ \\
& & & & & $\Delta t_{\text{prev}} = 0.050$s \\
\hline
\end{tabular}
\end{table}

For marks with meaningful locality (price, size, time), we structure the label space hierarchically. A field with $N$ 
fine bins is partitioned into $C$ coarse groups. For price, these groups typically represent "negative", "zero", and "positive" 
moves (or finer granularities like "small negative", "large negative"). Each fine bin $k$ maps to a coarse index $c(k)$ and a 
residual index $r(k)$. This structure informs both the model architecture and the loss function.

\subsection{Backbone}
For an input segment of length $L$, we embed each field and concatenate the embeddings. A learned projection maps the 
concatenated vector to the model dimension $d$. An additional projection of $\log(1+\Delta t_{i})$ injects inter-arrival 
information; cyclical encodings of absolute time-of-day are added when enabled. The resulting sequence is passed through $n$ 
Mamba layers, followed by LayerNorm and dropout.
\begin{figure}[t]
    \centering
    \includegraphics[width=0.9\linewidth]{figures/mamba_model.png}
    \caption{Detailed view of the Mamba History Model. The architecture ingests a sequence of derived features, processes 
    them through a stack of Mamba layers, and branches into a temporal head (for arrival times) and hierarchical mark heads 
    (for event attributes).}
    \label{fig:mamba-detail}
\end{figure}

\subsection{Temporal head: mixture of exponentials}
Given hidden states $h_i \in \mathbb{R}^{d}$, the temporal head outputs non-negative weights and rates
$(\alpha_{i,k}, \beta_{i,k})_{k=1}^K$ via linear projections and softplus activations. Using~\eqref{eq:moe-intensity}, 
the log-likelihood for an observed $\Delta t_i$ is
\begin{equation}
\ell_{\text{time}} = \log \lambda(\Delta t_i) - \Lambda(\Delta t_i),
\end{equation}
while censored observations (no arrival before a cap) contribute $-\Lambda(\Delta t_{\max})$. The probability of at least 
one arrival within a horizon $\tau$ is $1 - e^{-\Lambda(\tau)}$, used for calibration and downstream risk flags.

\subsection{Hierarchical mark heads}
For each mark field, the prediction is factorized into a coarse selection and a residual refinement:
\[
p(m) = p(c \mid h_i) \, p(r \mid c, h_i).
\]
The coarse probability $p(c \mid h_i)$ is obtained via a linear layer and softmax. To condition the residual prediction on the coarse choice, we compute a context vector $\mathbf{c}_{\text{ctx}}$ by aggregating learnable coarse embeddings $\mathbf{E}^{\text{coarse}}$ weighted by the predicted coarse probabilities (soft conditioning):
\[
\mathbf{c}_{\text{ctx}} = \sum_{j} p(j \mid h_i) \mathbf{E}^{\text{coarse}}_j.
\]
The residual logits are then computed from the concatenation $[h_i; \mathbf{c}_{\text{ctx}}]$ via a specialized linear layer that outputs logits for all possible $(c, r)$ pairs, reshaped and masked to valid combinations. This design allows the model to share information across the coarse group while refining the specific bin prediction.

\subsection{Sampling}
Sampling a next event involves drawing $\Delta t$ from the mixture via inverse transform (with Newton refinement) 
and sampling marks by first drawing a coarse bin and then a residual, mapping back to fine indices. This allows joint 
Monte Carlo scenarios for stress testing or simulation.


\section{Training setup}

\subsection{Data splits and bin fitting}
We fit quantile-based bin edges for size and time on the training portion (default 80\%), avoiding look-ahead. Price bins use a fixed centered window around the spread with two tails. Validation uses the remaining held-out segment (default 10\%), with the final 10\% reserved for future test reporting. Sliding windows of length $L$ and stride $s$ build sequences; the last position in each window is masked for mark supervision.

\subsection{Loss functions}
The total objective is a weighted sum:
\[
\mathcal{L} = \lambda_{\text{time}} \, \mathcal{L}_{\text{TPP}} + \sum_{f} \bigl(\lambda_{f}^{\text{coarse}} \mathcal{L}_{f}^{\text{coarse}} + \lambda_{f}^{\text{resid}} \mathcal{L}_{f}^{\text{resid}}\bigr),
\]
where $f$ ranges over marks (price, size, time, type, side, level).
\begin{itemize}[leftmargin=*,itemsep=0.25em]
    \item \textbf{Temporal loss} $\mathcal{L}_{\text{TPP}}$ is the negative log-likelihood from the mixture-of-exponentials head. Observed events use $\log \lambda - \Lambda$; censored events contribute $\Lambda(\Delta t_{\max})$.
    \item \textbf{Mark losses} use cross-entropy with neighbor-aware label smoothing. We construct a smoothing matrix $S$ based on the adjacency of bins. For a target bin $y$, the smoothed target distribution $q(k)$ is:
    \[
    q(k) = \begin{cases}
    1 - \epsilon & \text{if } k = y \\
    \epsilon / |N(y)| & \text{if } k \in N(y) \\
    0 & \text{otherwise}
    \end{cases}
    \]
    where $N(y)$ is the set of immediate neighbors of $y$ in the ordered bin sequence, and $\epsilon$ is a smoothing parameter. This is applied to both coarse and residual targets, penalizing "near misses" less than far-off errors.
\end{itemize}
Continuous targets that exceed a censoring cap are masked, aligning the supervision with the near-term horizon of interest.

\subsection{Optimization}
We use the AdamW optimizer with $\beta_1=0.9, \beta_2=0.999$, and weight decay of $0.1$. The learning rate follows a cosine decay schedule with a linear warmup phase. Specifically, for the first $W$ steps, the learning rate increases linearly to $\eta_{\text{peak}}$, then decays according to:
\[
\eta_t = \frac{\eta_{\text{peak}}}{2} \left(1 + \cos\left(\frac{\pi (t - W)}{T - W}\right)\right)
\]
where $T$ is the total number of steps. Gradient clipping (norm 1.0) stabilizes training. Dropout is applied to embeddings (0.15) and MLP blocks (0.32). Mixed precision (FP16/BF16) is enabled for efficiency. Key hyperparameters include mixture size $K=3$, horizon $\tau=0.75$s, and model dimension $d=512$.

\subsection{Inference and calibration}
At inference, $P(\Delta t \le \tau)$ provides a calibrated near-horizon arrival probability. Mark posteriors combine coarse and residual distributions. Reliability is monitored via Brier score and calibration bins; sampling utilities enable scenario analysis and downstream simulators.

\section{Evaluation plan}

We report both temporal and mark-centric metrics, emphasizing calibration for trading use-cases.
\begin{itemize}[leftmargin=*,itemsep=0.25em]
    \item \textbf{Temporal metrics}: negative log-likelihood of $\Delta t$, Brier score for $P(\Delta t \le \tau)$, expected calibration error (ECE), and coverage of calibration bins.
    \item \textbf{Mark metrics}: coarse and residual accuracy/entropy per field; joint mark accuracy; hierarchical perplexity when applicable.
    \item \textbf{Sampling diagnostics}: empirical mean and variance of sampled $\Delta t$; histograms of sampled coarse marks compared to validation frequencies.
\end{itemize}

Experiments will be added after completing broader evaluations across instruments and days. We will focus on (i) ablations over mixture size $K$, binning schemes, and smoothing strengths; (ii) the impact of time-of-day and soft binning; and (iii) robustness to regime shifts (open/close, news events).

\section{Related work}
\label{sec:related-work}

\paragraph{Neural temporal point processes.}
Recurrent and attention-based neural TPPs learn history-dependent intensities without hand-crafted kernels \citep{du2016recurrent,mei2017neural,shchur2021review,tan2021relaxing}. Many models trade off expressiveness and tractable integrals; basis expansions or mixtures (including exponentials) keep survival functions closed-form. Our approach uses a mixture-of-exponentials head to retain analytical likelihoods while delegating history modeling to a state-space backbone.

\paragraph{State-space sequence models.}
Selective state-space models such as Mamba \citep{gu2023mamba} provide linear-time sequence processing with competitive long-context capacity. Compared to transformers, they reduce quadratic cost and memory, making them attractive for long LOB streams. We combine Mamba with continuous-time conditioning to capture both inter-arrival structure and latent regime shifts (e.g., time-of-day).

\paragraph{Limit order book modeling.}
Deep models for LOB data have focused on price movement classification, depth forecasting, or point processes over order events. Prior work often uses Hawkes processes or RNN/transformer architectures with discrete time steps. By coupling a continuous-time head with hierarchical mark decoders, we aim to obtain calibrated arrival probabilities and structured mark distributions aligned with microstructure (spread-centered price bins, heavy-tailed sizes).

\section{Conclusion and next steps}

We presented a continuous-time LOB forecasting architecture that integrates a mixture-of-exponentials TPP head with hierarchical mark decoders on top of a Mamba state-space backbone. The design emphasizes calibrated near-horizon risk estimates and structured mark distributions while remaining computationally efficient for long streams.

Future work will extend the empirical study to multiple assets and regimes, explore robustness under distribution shift, and compare against transformer-based TPP baselines. Additional extensions include richer mark taxonomies (e.g., depth snapshots), adaptive binning under drift, and deployment-oriented latency profiling.


\bibliographystyle{alpha}
\bibliography{bib/refs.bib}

\end{document}
